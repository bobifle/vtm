\documentclass[10pt,a4paper,twoside,twocolumn,openany]{book}
% we'll need the old hline later
\let\oldhline\hline

\usepackage[bg-a4]{dnd} % Options: bg-a4, bg-letter, bg-full, bg-print, bg-none.
%\usepackage[english]{babel}
\usepackage[francais]{babel}
\usepackage[most]{tcolorbox}
\usepackage{caption}

\backgroundsetup{
scale=1,
angle=0,
opacity=1,
contents={\begin{tikzpicture}
\path (0,0) rectangle (\paperwidth,\paperheight);
\node[scope fading=west,inner sep=0pt,outer sep=0pt,anchor=north east] at(\paperwidth,\paperheight) {\includegraphics[height=\paperheight]{twilek_red.jpg}};
\end{tikzpicture}}
}


\begin{document}

\chapter{Sib Morh Koh Mustafar}

\tikz[remember picture,overlay] \node[opacity=1,inner sep=0pt] at (current page.center){\includegraphics[width=\paperwidth,height=\paperheight]{twilek_red.jpg}};

\chapter{Sib Morh Koh Mustafar}

\section{L'enfant de Mustfar}

Née sur la planète rouge, Sib a vécu toute sa jeunesse sur cette planète. Issue d'une famille
bourgeoise, son père est un ancien militaire décoré sensible à la force, reconverti en negociant.

Mais très tôt Sib est repérée comme une utilisatrice de la force, bien plus douée que son père. Elle intègre
un établissement spécialisé dans la formation de jedi. Ce n'est pas à proprement parlé une académie car elle n'est pas officiellement approuvée par le conseil jedi mais la bordure exterieure jouit d'une certaine autonomie et donc d'une certaine liberté.

\begin{commentbox}{}
Son nom: litéralement Sib Morh de Mustafar. Les habitants de cette planète sont souvent très fiers de prosperer dans un
environnement aussi hostile et il n'est pas rare qu'il mentionne leur plannète dans leur prore nom de famille.
\end{commentbox}

\section{Un potentiel mal exploité}

Sib est une adolescente rebelle, irrespectueuse et souvent feignante. Persuadée qu'elle possède un don hors du commun elle néglige sa formation et s'attache surtout à s'amuser et se divertir.

Elle entre réguièrement en conflit avec ses formateurs, mais les connections et l'influence locale de sa famille font qu'elle est rarement inquiétée. Jusqu'au jour ou agacé son père décide de la retirer de l'établissement et de la faire travailler pour l'entreprise familiale.

Mais une fois de plus Sib n'en fait qu'a sa tête et se revèle être une pière négociante.

\section{Au service de Doku}

Alors qu'elle faisait semblant de travailler, le transport de matière première qu'elle "accompagnait" (elle avait un faible pour l'un des conducteurs) se fait attaquer. Les braconneurs ne s'attendaient probablement pas a faire face a un apprenti jedi, et Sib se montra sans concession, tuant deux des assaillants et faisant fuir le reste de la bande à elle seule.   

Un droide de sécurité appartenant au destinataire du convoi fut témoin de la scène. Il en référa à son propriétaire, un certain Maître Doku. Celui-ci contacta la famille de Sid et proposa à son père de terminer sa formation.
Etrangement Sib accepta sans hesiter, quelque chose la fascinait chez Doku, elle lui trouvait un charisme hors \color{white} du commun malgré le fait qu'elle ne le connaissait que depuis quelques heures.

\section{La séparation}
Rapidement Doku s'apperçut du potentiel relativement commun de Sib. Cendant il décida de la conserver car elle était extremement dévouée, et utile. Elle acceptait sans rechigner des tâches que certains apprentis Jedi auraient pu discuter...

Cependant l'admiration totalement a sens unique de Sib envers Doku laissa
place à de l'amertume. En effet elle se rendi compte avec le temps que Doku
 ne la considérait pas réellement comme son apprenti.
Sa dévotion laissa place à de la rancune. Soucieux d'eviter un conflit imminent, Doku se debarassa de Sib en
l'assignant à un groupe de mercenaire venu sur Eriadu proteger une negociation importante.

\begin{commentbox}{Sib en quelques points}

\begin{itemize}
\item jolie et athlétique
\item charactérielle, indépendante
\item parfois méprisante envers les faibles
\item généreuse, dédiée et motivée, pour peu qu'elle apprécie quelqu'un
\end{itemize}

\end{commentbox}

\vspace{2cm}

\begin{figure}
\centering
\includegraphics[scale=0.4]{mustafar.png}
\caption*{\color{white}Mustafar, la planète rouge}
\end{figure}


\end{document}
